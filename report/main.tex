\documentclass{article}
\usepackage[utf8]{inputenc}
\usepackage[english]{babel}
\usepackage{graphicx}
\usepackage[hidelinks]{hyperref}
\usepackage{caption}
\usepackage{xcolor}
\usepackage{amsmath, amsthm, amssymb}

\usepackage{csquotes}
\usepackage{subcaption}
\usepackage{appendix}
\usepackage{listings}
\usepackage{lmodern}
\usepackage{color}
\usepackage{cancel}
\usepackage{float}
\usepackage{booktabs}


\lstdefinestyle{lfonts}{
  basicstyle   = \footnotesize\ttfamily,
  stringstyle  = \color{purple},
  keywordstyle = \color{magenta}\bfseries,
  numberstyle   =  \tiny\color{gray}
  commentstyle = \color{green}\scshape,
}
\lstdefinestyle{lnumbers}{
  numbers     = left,
  numberstyle = \tiny,
  numbersep   = 1em,
  firstnumber = 1,
  stepnumber  = 1,
}
\lstdefinestyle{llayout}{
  breaklines       = true,
  tabsize          = 2,
  columns          = flexible,
}
\lstdefinestyle{lgeometry}{
  xleftmargin      = 20pt,
  xrightmargin     = 0pt,
  frame            = tb,
  framesep         = \fboxsep,
  framexleftmargin = 20pt,
}
\lstdefinestyle{lgeneral}{
  style = lfonts,
  style = lnumbers,
  style = llayout,
  style = lgeometry,
}
\lstdefinestyle{c}{
    language = {c},
    style    = lgeneral,
}



\newcommand{\todo}[1]{{\color{blue}#1}}  % show to-do items in blue
\setlength{\parskip}{\baselineskip}
\usepackage[activate={true,nocompatibility},
            final,
            tracking=true,
            kerning=true,
            factor=1100,
            stretch=10,
            shrink=10]{microtype}
\hypersetup{
    pdftitle={assignment 1},
    % pdfpagemode=fullscreen,
}

\usepackage[a4paper,top=2cm,bottom=2cm,left=3cm,right=3cm,marginparwidth=1.75cm]{geometry}
\newcounter{questionnum} \setcounter{questionnum}{0}

\setlength{\parindent}{0pt}

\def\email#1{{\tt#1}}

\lstset{ % set the default style for code listings
	numbers=left,
	numberstyle=\scriptsize,
	numbersep=8pt,
	basicstyle=\scriptsize\ttfamily,
	keywordstyle=\color{blue},
	stringstyle=\color{red},
	commentstyle=\color{green!70!black},
	breaklines=true,
	frame=single,
	language=c,
	tabsize=4,
	showstringspaces=false
}


\begin{document}
\begin{titlepage}
  \begin{center}
    \includegraphics[width=0.6\textwidth]{figures/UU_logo.eps}
  \end{center}
  \vspace{3em}
  \begin{center}
    \Large Computational Science, Bridging Course
  \end{center}
  \vspace{5em}
  \begin{center}
    \Large Assignment 2
  \end{center}
  \vspace{10em}
  \begin{center}
    Yu Sheng Teh\\[0.5em]
    \today
  \end{center}
\end{titlepage}

\tableofcontents
\newpage

\textbf{All the codes to the solution written can be found from this GitHub:} \\
\url{https://github.com/vincent-teh/CS-assignment-2.git}

\section{Workout 1}%
\label{sec:Workout 1}
\subsection{Pendulum simulation in Python}%
\label{sub:Pendulum simulation in Python}
The given equations about the connected pendulums are a 2nd-order equation. Hence, the equations need to be converted to first order before solving using Python. 

\begin{align}
  \theta_1 &= a_1    \\
  \theta_1' &= a_1' = b_1 \\
  \theta_2 &= a_2  \label{eqn:q1:ode1} \\
  \theta_2' &= a_2' = b_2 \label{eqn:q1:ode2} \\
  \theta_1'' = b_1' &= -(\sin(a_1) + \alpha(a_1 - a_2)) \label{eqn:q1:ode3} \\
  \theta_2'' = b_2' &= -\sin(a_2) + \alpha(a_1 - a_2) \label{eqn:q1:ode4}
\end{align}
Equation \ref{eqn:q1:ode1}, \ref{eqn:q1:ode2}, \ref{eqn:q1:ode3}, \ref{eqn:q1:ode4} represent the systems of ODE present in the two pendulum problem. Using the ODE systems and calculate using Python,


\begin{figure}[H]
\centering
\includegraphics[width=0.8\textwidth]{figures/Pendulum_Motions_1.eps}
\caption{Pendulum simulation 1}
\label{fig:q1-pend-sim-1}
\end{figure}

\begin{figure}[H]
\centering
\includegraphics[width=0.8\textwidth]{figures/Pendulum_Motions_2.eps}
\caption{Pendulum simulation 2}
\label{fig:q1-pend-sim-2}
\end{figure}

\begin{figure}[H]
\centering
\includegraphics[width=0.8\textwidth]{figures/Pendulum_Motions_3.eps}
\caption{Pendulum simulation 3}
\label{fig:q1-pend-sim-3}
\end{figure}

Figure \ref{fig:q1-pend-sim-1}, \ref{fig:q1-pend-sim-2}, and \ref{fig:q1-pend-sim-3} shows the oscillation of the pendulum A and B across 10 seconds with the initial condition given.

\begin{figure}[H]
  \centering
  \includegraphics[width=0.8\textwidth]{figures/Pendulum-A.eps}
  \caption{Histogram of the position of Pendulum A at time T.}
  \label{fig:figures-Pendulum-A-eps}
\end{figure}

\begin{figure}[H]
  \centering
  \includegraphics[width=0.8\textwidth]{figures/Pendulum-B.eps}
  \caption{Histogram of the position of Pendulum B at time T.}
  \label{fig:figures-Pendulum-B-eps}
\end{figure}
Figure \ref{fig:figures-Pendulum-A-eps} and \ref{fig:figures-Pendulum-B-eps} shows the histogram of the final position of the both pendulums. The distribution of the position of the pendulums follows the central limit theorem and form the normal distribution to which the mean and standard deviation of the positions can be calculated. 


\begin{table}[H]
\centering
\caption{Statistical Data of A1(T) and A2(T)}
\label{tab:statistical data a1 a2}
\begin{tabular}{|l|r|}
\hline
\textbf{}       & \textbf{Value} \\
\hline
mean of A1(T)   & -0.7713        \\
std dev of A1(T) & 0.0191         \\
mean of A2(T)   & -0.7720        \\
std dev of A2(T) & 0.0194         \\
\hline
\end{tabular}
\end{table}
Table \ref{tab:statistical data a1 a2} shows the statistical data of the position of A1 and A2 at time $T=10$. The mean here represents the expectation of A1 and A2 at time T.

The probability of $0 \le A1(T),\ A2(T) < \pi \approx 0$. 

The probability of $-\pi/4 \le A1(T),\ A2(T) < 0 \approx 76.98\%,\ 75.46\%$ respectively. 

Since the given ODE system is a non-stiff problem, RK45 as an explicit method was chosen due to RK45's computational efficiency.


\section{Workout 2}%
\label{sec:Workout 2}
\subsection{1D FEM with the Linear Diffusion Equation.}%
\label{sub:1D FEM with the Linear Diffusion Equation.}
Given the following equation:
\begin{equation}
  \frac{\partial}{\partial t} u(x,t) = \frac{\partial}{\partial x} \left( a(x) \frac{\partial u(x,t)}{\partial x} \right), \quad a \leq x \leq b, \quad 0 < t \leq t_{final} \label{eqn:w2:original eqn}
\end{equation}
First, apply the weak form to equation \ref{eqn:w2:original eqn}:
\begin{align}
  \int_{\Omega} \frac{\partial}{\partial t} u(x,t) \, v \, dx = \int_{\Omega} \frac{\partial}{\partial x} \left( a(x) \frac{\partial u(x,t)}{\partial x} \right) v \, dx \\
  \int_{\Omega} \frac{\partial}{\partial t} u(x,t) \, v \, dx = - \int_{\Omega}  a(x) \frac{\partial u(x,t)}{\partial x} \frac{\partial v}{\partial x} \, dx
\end{align}
where $\Omega$ is the boundary of interest of the integral from $a$ to $b$, $v(x)$ is the test function.
\begin{align}
  \text{let } u = & \sum_j u_j (t) \phi(x) ,\ v(x) = \phi(x) \\
  \int_{\Omega} \frac{\partial}{\partial t} \left( \sum_{i} u_{i}(t) \phi_{i}(x) \right) dx &= -\int_{\Omega} a(x) \frac{\partial}{\partial x} \left( \sum_{j} u_{j}(t) \phi_{j}(x) \right) \frac{\partial \phi(x)}{\partial x} \, dx \\
  \sum_j \frac{\partial u_j(t)}{\partial t} \int_{\Omega} \phi_j(x) \phi_i(x) \, dx &= - \sum_j u_j(t) \int_{\Omega} a(x) \frac{\partial}{\partial x} \phi_j(x) \frac{\partial}{\partial x} \phi_i(x) \, dx \\
  \text{let } M = \int_{\Omega} \phi_j(x) \phi_i(x)& \, dx ,\ S = -\int_{\Omega} a(x) \frac{\partial}{\partial x} \phi_j(x) \frac{\partial}{\partial x} \phi_i(x) \, dx \\
  M \frac{d {u(t)}}{d {t}} &= S u(t), \, 0 < t \le t_{\text{final}} \label{eqn:w2:derived 1}
\end{align}
Hence, proved.

$u(t)$ is the coefficient function with respect to time that describes the height of each point of the mesh at time t. Since the number of mesh is describe by $h$, the distance between each point will be $h = \frac{a - b}{N}$ and the span of the mesh is $\{a, a+h, a+2h, a+3h, \dots b\}$.

\begin{align}
  u(0) &= f(x) \\
      &= \begin{bmatrix}
        u_1(0) \\ u_2(0) \\ \vdots
      \end{bmatrix} \\
      &= \begin{bmatrix}
        f(a+h) \\ f(a+2h) \\ \vdots \\ f(b-h)
      \end{bmatrix}
\end{align}
In other words, $u_i = f(a + ih)$.




\subsection{Proved of M is a +ve definite matrix}%
\label{sub:Proved of M is a +ve definite matrix}
The Sylvester's criteria states that any nxn Hermitian matrix is positive definite if and only if all of its upper corner matrix has positive determinant. Given that the mass matrix is a $n \times n$ tridiagonal matrix of the collection of $[[2, 1], [1, 2]]$ matrix:
\begin{align}
  \det(M_2) &= \det \begin{bmatrix}
    2 & 1 \\
    1 & 2
  \end{bmatrix} \\
   &= 4 \\
    \det(M_3) &= \det \begin{bmatrix}
      2 & 1 \\
      1 & 4 & 1\\
        & 1 & 2
    \end{bmatrix} \\
  &= 2 \det \begin{bmatrix}
    4 & 1 \\
    1 & 2
  \end{bmatrix} - \det \begin{vmatrix}
    1 & 1 \\ 0 & 2
  \end{vmatrix} \\
  &= 14 - 1 = 13 \\
  \det(M_4) &= \begin{vmatrix}
    2 & 1 \\ 1 & 4 & 1 \\
      & 1 & 4 & 1 \\ & & 1 & 2
  \end{vmatrix} \\
  &=  2 \begin{vmatrix}
    4 & 1 \\ 1 & 4 & 1 \\ & 1 & 2
  \end{vmatrix} - \begin{vmatrix}
    1 & 1 \\ & 4 & 1 \\ & 1 & 2
  \end{vmatrix} \\
  &= 7 \begin{vmatrix}
    4 & 1 \\ 1 & 2
  \end{vmatrix} - \begin{vmatrix}
    1 & 1 \\ 0 & 2
  \end{vmatrix} + \begin{vmatrix}
    0 & 1 \\ 0 & 2
  \end{vmatrix} \\
  &= 47 \\
\end{align}
\begin{align}
  \det (M_5) &= \begin{vmatrix}
      2 & 1 \\
      1 & 4 & 1\\
        & 1 & 4 & 1 \\
        & & 1 & 4 & 1 \\
        & & & 1 & 2
    \end{vmatrix} \\
  &= 2 \begin{vmatrix}
    4 & 1 \\ 1 & 4 & 1 \\
      & 1 & 4 & 1 \\ & & 1 & 2
  \end{vmatrix} - \begin{vmatrix}
    1 & 1 \\
      & 4 & 1 \\
      & 1 & 4 & 1 \\
      & & 1 & 2
  \end{vmatrix} \\
  &= 2 \cdot \left( 4 \begin{vmatrix}
    4 & 1 \\ 1 & 4 & 1 \\ & 1 & 2
  \end{vmatrix} - \begin{vmatrix}
    1 & 1 \\ & 4 & 1 \\ & 1 & 2
  \end{vmatrix} \right) - \begin{vmatrix}
    4 & 1 \\ 1 & 4 & 1 \\ & 1 & 2
  \end{vmatrix} + \begin{vmatrix}
    0 & 1 & \\ 0 & 4 & 1 \\ 0 & 1 & 2
  \end{vmatrix} \\
  &= 2 \left( 
    4 \cdot 3 \begin{vmatrix}
      4 & 1 \\ 1 & 2
    \end{vmatrix} - \begin{vmatrix}
      1 & 1 \\ 0 & 2
    \end{vmatrix}
    \right) - 4 \begin{vmatrix}
      4 & 1 \\ 1 & 2
    \end{vmatrix} + \begin{vmatrix}
      1 & 1 \\ 0 & 2
    \end{vmatrix} \\
  &= 20 \begin{vmatrix}
    4 & 1 \\ 1 & 2
  \end{vmatrix} - \begin{vmatrix}
    1 & 1 \\ 0 & 2
  \end{vmatrix} \\
  &= c * 7 - 2
\end{align}
Since $c$ always positive as the element of the matrix M are all positive and the value of the second diagonal is always smaller than the center diagonal element, the determinant of $M_n$ is also always positive, the tridiagonal matrix follows the Sylvester's criteria and hence is a positive definite matrix.



\subsection{Solving using the std ODE solver}%
\label{sub:Solving using the std ODE solver}
Explicit Euler's method:
\begin{align}
  y_{k+1} &= y_k + hf(t_k, y_k) \\
  u(t_{k+1}) &= u(t_k) + h \frac{\partial u(t_k)}{\partial t} \\
  u(t_{k+1}) &= u(t_k) + h M^{-1} S u(t_k) \\
  M u(t_{k+1}) &= M u(t_k) + h S u(t_k) \\
  M u(t_{k+1}) &= (M + hS) u(t_k) \label{eqn:w2:derived-explicit-euler}
\end{align}

Implicit Euler Formula:
\begin{align}
  u(t_{k+1}) &= u(t_k) + h \frac{\partial u(t_{k+1})}{\partial t} \\
             &= u(t_k) + h M^{-1} b(t_{k+1}) \\
  M u(t_{k+1}) &= M u(t_k) + hb(t_{k+1}) \\
  M u(t_{k+1}) &= M u(t_k) + h S u(t_{k+1}) \\
  (M - h S) u(t_{k+1}) &= M u(t_k) \label{eqn:w2:derive-implicit-euler}
\end{align}
Solve equation \ref{eqn:w2:derive-implicit-euler} to obtain the updated $u(t_{k+1})$ at each time steps.

Trapezoidal method:
\begin{align}
  u(t_{k+1}) &= u(t) + \frac{h}{2} \left(u'(t) + u'(t_{k+1}) \right) \\
             &= u(t) + \frac{h}{2} \left( M^{-1} S u(t) + M^{-1} S u(t_{k+1}) \right) \\
  M u(t_{k+1}) &= M u(t_k) + \frac{h}{2} S u(t) + \frac{h}{2} S u(t_{k+1}) \\
  \left(M - \frac{h}{2}S \right) u(t_{k+1}) &= \left( M + \frac{h}{2}S \right) u(t) \label{eqn:w2:derived-trapezoidal-method}
\end{align}
Similar to the Implicit Euler Method, the Trapezoidal Method also produce a set of linear equation that can be solved using standard linear system solver.

In Python, use \texttt{numpy.linalg.solve} to create a system of linear equation solver with respect to $\frac{\partial u(t)}{\partial t}$. The built-in Numpy linear system solver will implement the LU decomposition. Then, apply the system of equation into \texttt{scipy.integrate.solve\_ivp}.


\subsection{Stability Analysis of the ODE methods on the FEM of the Heat Equation.}%
\label{sub:Stability Analysis of the ODE methods on the FEM of the Heat Equation.}
% The Courant–Friedrichs–Lewy condition defines the heat equation is stable if:
% \begin{align}
%   \max |u(t_{k+1})| \le \max |u(t_k)|
% \end{align}
% To satisfy the CFL condition using Explicit Euler's method,
% \begin{align}
%   k \le \frac{h^2}{2}
% \end{align}
% Where k is the time step and h is the size of the interval of the mesh.
Given the derived Explicit Euler equation
\begin{align}
  u(t_{k+1}) &= u(t_k) + kM^{-1}S u(t_k) \\
  \text{Let } A &= M^{-1}S
\end{align}
The matrix A can be rewritten as:
\begin{align}
A = \frac{1}{h^2}\begin{bmatrix}
  -2 & 1 \\
  1 & -2 & 1 \\
   & 1 & 2 & -1 \\
   & & \ddots & \ddots & \ddots \\
   & & & 1 & -2 & 1 \\
   & & & & 1 & -2
\end{bmatrix}
\end{align}
The system is stable if and only if 
\begin{align}
|1 + k \lambda| \le 1
\end{align}
Where $\lambda$ is the maximum eigenvalues of A and the Eigenvalues of A can be computed as:
\begin{align}
  \text{Eigenvalue of }A = \frac{-4}{h^2}\sin^2( \frac{\pi}{2} l h) \\
\end{align}
We can obtained the absolute stability region with the following equation:
\begin{align}
\frac{2}{-\lambda_m} = \frac{h^2}{2 \sin^2 \left( \frac{\pi}{2} m h \right)}
\end{align}
since the denominator $-1 \le \sin( \frac{\pi}{2}m h) \le 1$, the maximum eigenvalue of A is obtained simply by equating the denominator with 1. Then,
\begin{align}
|1 + k\lambda| \le 1 \\
|1 + k \frac{4}{h^2}| \le 1 \\
k \frac{4}{h^2} \le 2 \\
k \le \frac{h^2}{2}
\end{align}


Implicit method on the other hand has unconditional stability criteria, implying that any choice of h will result in stable system.





\subsection{Implementation of the FEM in Python.}%
\label{sub:Implementation of the FEM in Python.}
\begin{figure}[H]
  \centering
  \includegraphics[width=0.8\textwidth]{figures/Explicit_Euler.eps}
  \caption{Explicit Euler method.}
  \label{fig:figures-Explicit_Euler-eps}
\end{figure}

\begin{figure}[H]
  \centering
  \includegraphics[width=0.8\textwidth]{figures/Implicit_Euler.eps}
  \caption{Implicit Euler method.}
  \label{fig:figures-Implicit_Euler-eps}
\end{figure}

\begin{figure}[H]
  \centering
  \includegraphics[width=0.8\textwidth]{figures/Trapezoidal.eps}
  \caption{Trapezoidal Method.}
  \label{fig:figures-Trapezoidal-eps}
\end{figure}
Figure \ref{fig:figures-Explicit_Euler-eps}, \ref{fig:figures-Implicit_Euler-eps} and
\ref{fig:figures-Trapezoidal-eps} show that all three methods able to converge to a reasonable
solution. The $N$ of the mesh is set to 7 while the steps taken, $k$ is set to 100. The code was
created by first generate the $u_0 = sin(x)$ vector. Then, solve the $Ax = b$ equation derived 
in \ref{eqn:w2:derived-explicit-euler}, \ref{eqn:w2:derive-implicit-euler} and 
\ref{eqn:w2:derived-trapezoidal-method} iteratively for $k$ steps.


\begin{figure}[H]
  \centering
  \includegraphics[width=0.8\textwidth]{figures/mesh-analysis-Explicit_Euler.eps}
  \caption{Effect of decreasing the mesh sizes of the Explicit Euler method.}
  \label{fig:figures-mesh-analysis-Explicit_Euler-eps}
\end{figure}

\begin{figure}[H]
  \centering
  \includegraphics[width=0.8\textwidth]{figures/mesh-analysis-Implicit_Euler.eps}
  \caption{Effect of decreasing the mesh sizes of the Implicit Euler method.}
  \label{fig:figures-mesh-analysis-Implicit_Euler-eps}
\end{figure}

\begin{figure}[H]
  \centering
  \includegraphics[width=0.8\textwidth]{figures/mesh-analysis-Trapezoidal.eps}
  \caption{Effect of decreasing the mesh sizes of the Trapezoidal method.}
  \label{fig:figures-mesh-analysis-Trapezoidal-eps}
\end{figure}
The choice of $k$ for the figures plotted are 10000 and the sample size taken are 
$[5, 7, 10, 20, 40, 80, 100]$. Based on figures \ref{fig:figures-mesh-analysis-Explicit_Euler-eps}, \ref{fig:figures-mesh-analysis-Implicit_Euler-eps} and \ref{fig:figures-mesh-analysis-Trapezoidal-eps},
as the size of the mesh decreases, the error of the solution decreases as well. This is because
the error rate of the FEM, $||u - u_h|| \le Ch^p$ where $C$ is a constant. As $h$ becomes 
smaller, the approximation converge to the actual solution. Here, $k=10000$ was chosen to 
eliminate the possibility of the Explicit Euler Method becomes unstable numerically. On the 
other hand, the choice of the $k$ does not affect much on the numerical performance of the 
Implicit methods, but rather have impact on the runtime performance, which in this case is 
negligible.



\begin{figure}[H]
  \centering
  \includegraphics[width=0.8\textwidth]{figures/time-analysis-Explicit_Euler.eps}
  \caption{Effect of changing the step size on the Explicit Euler method.}
  \label{fig:figures-time-analysis-Explicit_Euler-eps}
\end{figure}

\begin{figure}[H]
  \centering
  \includegraphics[width=0.8\textwidth]{figures/time-analysis-Implicit_Euler.eps}
  \caption{Effect of changing the step size on the Implicit Euler method.}
  \label{fig:figures-time-analysis-Implicit_Euler-eps}
\end{figure}

\begin{figure}[H]
  \centering
  \includegraphics[width=0.8\textwidth]{figures/time-analysis-Trapezoidal.eps}
  \caption{Effect of changing the step size on the Trapezoidal method.}
  \label{fig:figures-time-analysis-Trapezoidal-eps}
\end{figure}

Figure \ref{fig:figures-time-analysis-Explicit_Euler-eps}, \ref{fig:figures-time-analysis-Implicit_Euler-eps} and \ref{fig:figures-time-analysis-Trapezoidal-eps}
show the ability of the methods to converge. The number of steps taken by the Explicit Euler
method is $[1000, 1300, 1500, 2000, 5000, 8000, 10000, 100000]$ while the step size of the
Implicit Method range from 10 to 3000. As the number of step size increases, the error of the
method decreases. In particular, the trapezoidal method converge fastest outshone the other two
algorithm. This is because the trapezoidal method utilized both the information of the current
step and the predicted next step, resulting in a more efficient updating steps. On the other hand,
the Explicit Euler method requires significantly large step size as the method demand strict
convergence criteria of $k \ge 1800$ for $N = 30$.






% \begin{figure}[H]
%   \centering
%   \includegraphics[width=0.8\textwidth]{figures/mesh-analysis-total.eps}
%   \caption{Effect of changing the mesh size on the accuracy across the 3 methods.}
%   \label{fig:figures-mesh-analysis-total-eps}
% \end{figure}
%
% Figure \ref{fig:figures-mesh-analysis-total-eps} shows as the mesh size increases, the accuracy 
% of the Explicit Euler method improves while the Implicit Euler and Trapezoidal methods have no 
% significant changes in their accuracy. This is because the Explicit Euler stability criteria are
% bounded by both the mesh size and the time step size. The larger the size of the mesh, the smaller the time step required. By increasing the size of the k to 20, figure \ref{fig:figures-mesh-analysis-total-eps} shows only the largest mesh reaches convergence despite the step size has doubled. This indicates the need of using  the large number of steps to achieve convergence, which
% can be computational expensive as the number of mesh and dimension increases.
%
% \begin{figure}[H]
%   \centering
%   \includegraphics[width=0.8\textwidth]{figures/mesh-analysis-total-converge.eps}
%   \caption{Explicit Euler method converge only for the largest mesh size for k=20.}
%   \label{fig:figures-mesh-analysis-total-converge-eps}
% \end{figure}
%
%
% \begin{figure}[H]
%   \centering
%   \includegraphics[width=0.8\textwidth]{figures/step-analysis-total.eps}
%   \caption{Effect of changing the step size on the error across the 3 methods.}
%   \label{fig:figures-step-analysis-total-eps}
% \end{figure}
% Figure \ref{fig:figures-step-analysis-total-eps} shows again only the Explicit Euler's method did not reach convergence at all the choice of the step sizes. Also noticed that even the implicit methods did not show improvement in accuracy with the constant time steps despite increasing the number of steps taken.
%
% \begin{figure}[H]
%   \centering
%   \includegraphics[width=0.8\textwidth]{figures/both-mesh-step.eps}
%   \caption{Increase both the Mesh and Step on Trapezoidal Method.}
%   \label{fig:figures-both-mesh-step-eps}
% \end{figure}
% Interestingly, increasing both the number of meshes and steps by applying meshes=[5, 6, 7, 8, 100] and steps=[5, 10, 20, 40, 100000] did not yield any improvement as shown in figure \ref{fig:figures-both-mesh-step-eps}. Thus, it is likely that the computation limit has been reached. One quick way to improve the performance is to use a higher accuracy data type, including a larger floating point to compensate higher accuracy.
%

\newpage
\section{Appendix: GitHub Link}%
\label{sec:Appendix: GitHub Link}
\url{https://github.com/vincent-teh/CS-assignment-2.git}

\end{document}

