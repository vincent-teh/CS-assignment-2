\documentclass{article}
\usepackage[utf8]{inputenc}
\usepackage[english]{babel}
\usepackage{graphicx}
\usepackage[hidelinks]{hyperref}
\usepackage{caption}
\usepackage{xcolor}
\usepackage{amsmath, amsthm, amssymb}

\usepackage{csquotes}
\usepackage{subcaption}
\usepackage{appendix}
\usepackage{listings}
\usepackage{lmodern}
\usepackage{color}
\usepackage{cancel}
\usepackage{float}
\usepackage{booktabs}


\lstdefinestyle{lfonts}{
  basicstyle   = \footnotesize\ttfamily,
  stringstyle  = \color{purple},
  keywordstyle = \color{magenta}\bfseries,
  numberstyle   =  \tiny\color{gray}
  commentstyle = \color{green}\scshape,
}
\lstdefinestyle{lnumbers}{
  numbers     = left,
  numberstyle = \tiny,
  numbersep   = 1em,
  firstnumber = 1,
  stepnumber  = 1,
}
\lstdefinestyle{llayout}{
  breaklines       = true,
  tabsize          = 2,
  columns          = flexible,
}
\lstdefinestyle{lgeometry}{
  xleftmargin      = 20pt,
  xrightmargin     = 0pt,
  frame            = tb,
  framesep         = \fboxsep,
  framexleftmargin = 20pt,
}
\lstdefinestyle{lgeneral}{
  style = lfonts,
  style = lnumbers,
  style = llayout,
  style = lgeometry,
}
\lstdefinestyle{c}{
    language = {c},
    style    = lgeneral,
}



\newcommand{\todo}[1]{{\color{blue}#1}}  % show to-do items in blue
\setlength{\parskip}{\baselineskip}
\usepackage[activate={true,nocompatibility},
            final,
            tracking=true,
            kerning=true,
            factor=1100,
            stretch=10,
            shrink=10]{microtype}
\hypersetup{
    pdftitle={assignment 1},
    % pdfpagemode=fullscreen,
}

\usepackage[a4paper,top=2cm,bottom=2cm,left=3cm,right=3cm,marginparwidth=1.75cm]{geometry}
\newcounter{questionnum} \setcounter{questionnum}{0}

\setlength{\parindent}{0pt}

\def\email#1{{\tt#1}}

\lstset{ % set the default style for code listings
	numbers=left,
	numberstyle=\scriptsize,
	numbersep=8pt,
	basicstyle=\scriptsize\ttfamily,
	keywordstyle=\color{blue},
	stringstyle=\color{red},
	commentstyle=\color{green!70!black},
	breaklines=true,
	frame=single,
	language=c,
	tabsize=4,
	showstringspaces=false
}


\begin{document}
\begin{titlepage}
  \begin{center}
    \includegraphics[width=0.6\textwidth]{figures/UU_logo.eps}
  \end{center}
  \vspace{3em}
  \begin{center}
    \Large Computational Science, Bridging Course
  \end{center}
  \vspace{5em}
  \begin{center}
    \Large Assignment 2
  \end{center}
  \vspace{10em}
  \begin{center}
    Teh Yu Sheng \\[0.5em]
    \today
  \end{center}
\end{titlepage}


\section{Workout 1}%
\label{sec:Workout 1}
\subsection{Pendulum simulation in Python}%
\label{sub:Pendulum simulation in Python}
The given equations about the connected pendulums are a 2nd-order equation. Hence, the equations need to be converted to first order before solving using Python. 

\begin{align}
  \theta_1 &= a_1    \\
  \theta_1' &= a_1' = b_1 \\
  \theta_2 &= a_2  \label{eqn:q1:ode1} \\
  \theta_2' &= a_2' = b_2 \label{eqn:q1:ode2} \\
  \theta_1'' = b_1' &= -(\sin(a_1) + \alpha(a_1 - a_2)) \label{eqn:q1:ode3} \\
  \theta_2'' = b_2' &= -\sin(a_2) + \alpha(a_1 - a_2) \label{eqn:q1:ode4}
\end{align}
Equation \ref{eqn:q1:ode1}, \ref{eqn:q1:ode2}, \ref{eqn:q1:ode3}, \ref{eqn:q1:ode4} represent the systems of ODE present in the two pendulum problem. Using the ODE systems and calculate using Python,


\begin{figure}[H]
\centering
\includegraphics[width=0.8\textwidth]{figures/Pendulum_Motions_1.eps}
\caption{Pendulum simulation 1}
\label{fig:q1-pend-sim-1}
\end{figure}

\begin{figure}[H]
\centering
\includegraphics[width=0.8\textwidth]{figures/Pendulum_Motions_2.eps}
\caption{Pendulum simulation 2}
\label{fig:q1-pend-sim-2}
\end{figure}

\begin{figure}[H]
\centering
\includegraphics[width=0.8\textwidth]{figures/Pendulum_Motions_3.eps}
\caption{Pendulum simulation 3}
\label{fig:q1-pend-sim-3}
\end{figure}


\begin{figure}[H]
  \centering
  \includegraphics[width=0.8\textwidth]{figures/Pendulum-A.eps}
  \caption{Histogram of the position of Pendulum A at time T.}
  \label{fig:figures-Pendulum-A-eps}
\end{figure}

\begin{figure}[H]
  \centering
  \includegraphics[width=0.8\textwidth]{figures/Pendulum-B.eps}
  \caption{Historgram of the position of Pendulum B at time T.}
  \label{fig:figures-Pendulum-B-eps}
\end{figure}


\begin{table}[H]
\centering
\caption{Statistical Data of A1(T) and A2(T)}
\begin{tabular}{|l|r|}
\hline
\textbf{}       & \textbf{Value} \\
\hline
mean of A1(T)   & -0.7713        \\
std dev of A1(T) & 0.0191         \\
mean of A2(T)   & -0.7720        \\
std dev of A2(T) & 0.0194         \\
\hline
\end{tabular}
\end{table}

\end{document}

